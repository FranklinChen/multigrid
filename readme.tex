\documentclass[titlepage,twoside]{article}

\usepackage{times}
\usepackage{latexsym}

\pagestyle{myheadings}
\markright{\texttt{CS 859E}: Assignment 2 (Franklin Chen)}

\author{Franklin Chen}
\title{15-859E Assignment 2: Multigrid}

\date{October 21, 1998}

\begin{document}

\maketitle

\tableofcontents
\listoffigures

\clearpage
\section{Equation}
\label{sec:equation}

I used the \emph{Poisson} equation in 2d:
\begin{displaymath}
  \nabla^2\mathbf{u} = \mathbf{f}
\end{displaymath}

This was suggested in the assignment, and was also discussed in
Briggs.


\subsection{Source}
\label{sec:source}

I tried some sample $\mathbf{f}$ for the equation.  One was
$\mathbf{f} = \mathbf{0}$ (Laplace).  I settled on another constant,
just for the heck of it.

I did not provide a user interface for plugging in various f, though
that would be easy, e.g., parse some stream of numbers from a file.


\subsection{Boundary values}
\label{sec:boundary-values}

The \emph{Dirichlet boundary values} I chose were to take the 2d unit
square as the boundaries, and have all sides be $0$ except a portion
of sine wave on one side, in order to get a simple but not completely
trivial solution surface.







\clearpage
\section{Implementation language}
\label{sec:language}

I implemented everything in \emph{Objective Caml} (O'Caml), a dialect
of ML.  I hadn't done much with this dialect, but figured now was as
good a time as any to experiment with it, instead of using
\emph{Standard ML}.

Since O'Caml supports arrays of unboxed floating point numbers, this
gave a fighting chance of not being too inefficient, though I suspect
that I could write C or C++ code that does significantly better.

Because I lacked time, I did not experiment with using a language with
high level support of arrays, such as ZPL, SAC, or FiSH.  I am
interested in trying this at some point.

Note: I use arrays of arrays for matrices.  This obviously misses out
on the memory locality and opportunities for good caching and loop
unrolling possible when using a block of contiguous memory as a
matrix (as the \texttt{svl} library in C++ attempts to do, as does the
SML/NJ library for Standard ML).

My first implementation was written purely functionally, with no side
effects in the computation, e.g., arrays were never modified.  I
reimplemented it to use mutation, and the speedup wasn't all that
large, actually.  I stuck to the mutative version though.

My primary source of equations was Briggs.  I also looked at
\emph{Numerical Recipes in C} earlier, did not consult it while in the
process of writing my own, since I first wrote all my code purely
functionally, then converted it in steps to use mutation.



\clearpage
\section{Platform}
\label{sec:platform}

I developed my program on my home PC, a Pentium 200 with 32 MB RAM
running Red Hat Linux 5.1.  I did some runs on my office PC, a Pentium
II 400 with 128 MB RAM, and performance appeared to be around four
times greater.



\clearpage
\section{Algorithms}

\subsection{Relaxation}
\label{sec:relaxation}

I implemented both \emph{Gauss-Seidel} (red-black) and \emph{Jacobi},
both of them with optional weighting ($\omega$) for overrelaxation.

I made no effort to make these crucial steps fast.  A lot of
interesting cache-related optimizations could be performed, but I
could only do that if I were using something as low level as C.


\subsection{Interpolation}
\label{sec:interpolation}

I used 2d linear interpolation, as suggested by Briggs.


\subsection{Restriction}

I used full-weighted 2d restriction, as suggested by Briggs.


\subsection{Multigrid}

I implemented the $\mu$-cycle generalization of the V-cycle, with fixed
numbers of iterations, specified for the duration of a run from the
command line.


\clearpage
\section{Command line arguments}
\label{sec:command-line}

Many parameters can be set for the duration of a run by means of
command line arguments.

\begin{verbatim}
Usage:
  -escape Escape from algorithm when relative residual small?
        Default = false
  -debug Debugging level
        Default = 0
        0 = no debug
        1 = output initial and final states
        2 = output after every relaxation step
        3 = output after each read and black step of Gauss-Seidel
        4 = output full multigrid residual and its restriction
  -nu0 Number of mu-cycles for each phase of full multigrid
        Default = 1
  -nu1 Number of pre-cycle relaxation steps in mu-cycle
        Default = 1
  -nu2 Number of post-cycle relaxation steps in mu-cycle
        Default = 1
  -mu Number of recursive calls in a mu-cycle
        Default = 1
  -coarse Size of coarsest grid base case (2^k+1)
        Default = 3
  -size Size of finest grid base case (2^k+1)
        Default = 9
  -omega Weight for relaxation method (in (0, 1])
        Default = 0.666666666667
  -relax Relaxation method (gs or jacobi)
        Default = gs
        gs = Gauss-Seidel
        jacobi = Jacobi
  -multigrid Overall method
        Default = full
        full = full multigrid
        mucycle = mu-cycle
        gs = Gauss-Seidel
        jacobi = Jacobi

  -iterations Number of iterations of relaxation
        Default = 1000
        (Only applicable to -multigrid gs|jacobi)
\end{verbatim}





\clearpage
\section{Timing, accuracy, speed}
\label{sec:timing}

Roughly, I was able to get full multigrid running to the desired
accuracy suggested (relative residual $< 10^{-6}$) on a $256 \times 256$
grid in about $9$ seconds on my home PC, and about $2$ seconds on my
office PC.

Attempts to get fine-grained data for algorithms other than full
multigrid were not very successful, because for larger $n$ things
would take forever, and furthermore, not achieve good accuracy.

I generated a bunch of runs by trying to guess parameters.  I tried to
be more scientific by adding code to detect when the relative residual
was small enough, but this expensive computation slowed things down
significantly.  I didn't try to just periodically do the check
instead.


\section{Visual presentations}
\label{sec:visual}

%graph

%fmc didn't fit gs?
\subsection{Plot}
\label{sec:plot}

In Figure \ref{all three}, we show the results for
\begin{itemize}
\item Weighted red-black Gauss-Seidel, 1300 iterations (note last two
  grid sizes did not complete in a reasonable time).
\item $\mu$-cycle, actually just a V-cycle, with 500 iterations pre and
  post.
\item Full multigrid, with parameters 4, 2, 2.
\end{itemize}

\begin{figure}
  \begin{center}
    % GNUPLOT: LaTeX picture
\setlength{\unitlength}{0.240900pt}
\ifx\plotpoint\undefined\newsavebox{\plotpoint}\fi
\sbox{\plotpoint}{\rule[-0.200pt]{0.400pt}{0.400pt}}%
\begin{picture}(1500,900)(0,0)
\font\gnuplot=cmr10 at 10pt
\gnuplot
\sbox{\plotpoint}{\rule[-0.200pt]{0.400pt}{0.400pt}}%
\put(220.0,113.0){\rule[-0.200pt]{4.818pt}{0.400pt}}
\put(198,113){\makebox(0,0)[r]{$0.015625$}}
\put(1416.0,113.0){\rule[-0.200pt]{4.818pt}{0.400pt}}
\put(220.0,203.0){\rule[-0.200pt]{4.818pt}{0.400pt}}
\put(198,203){\makebox(0,0)[r]{$0.0625$}}
\put(1416.0,203.0){\rule[-0.200pt]{4.818pt}{0.400pt}}
\put(220.0,293.0){\rule[-0.200pt]{4.818pt}{0.400pt}}
\put(198,293){\makebox(0,0)[r]{$0.25$}}
\put(1416.0,293.0){\rule[-0.200pt]{4.818pt}{0.400pt}}
\put(220.0,383.0){\rule[-0.200pt]{4.818pt}{0.400pt}}
\put(198,383){\makebox(0,0)[r]{$1$}}
\put(1416.0,383.0){\rule[-0.200pt]{4.818pt}{0.400pt}}
\put(220.0,473.0){\rule[-0.200pt]{4.818pt}{0.400pt}}
\put(198,473){\makebox(0,0)[r]{$4$}}
\put(1416.0,473.0){\rule[-0.200pt]{4.818pt}{0.400pt}}
\put(220.0,562.0){\rule[-0.200pt]{4.818pt}{0.400pt}}
\put(198,562){\makebox(0,0)[r]{$16$}}
\put(1416.0,562.0){\rule[-0.200pt]{4.818pt}{0.400pt}}
\put(220.0,652.0){\rule[-0.200pt]{4.818pt}{0.400pt}}
\put(198,652){\makebox(0,0)[r]{$64$}}
\put(1416.0,652.0){\rule[-0.200pt]{4.818pt}{0.400pt}}
\put(220.0,742.0){\rule[-0.200pt]{4.818pt}{0.400pt}}
\put(198,742){\makebox(0,0)[r]{$256$}}
\put(1416.0,742.0){\rule[-0.200pt]{4.818pt}{0.400pt}}
\put(220.0,832.0){\rule[-0.200pt]{4.818pt}{0.400pt}}
\put(198,832){\makebox(0,0)[r]{$1024$}}
\put(1416.0,832.0){\rule[-0.200pt]{4.818pt}{0.400pt}}
\put(220.0,113.0){\rule[-0.200pt]{0.400pt}{4.818pt}}
\put(220,68){\makebox(0,0){$8$}}
\put(220.0,812.0){\rule[-0.200pt]{0.400pt}{4.818pt}}
\put(423.0,113.0){\rule[-0.200pt]{0.400pt}{4.818pt}}
\put(423,68){\makebox(0,0){$16$}}
\put(423.0,812.0){\rule[-0.200pt]{0.400pt}{4.818pt}}
\put(625.0,113.0){\rule[-0.200pt]{0.400pt}{4.818pt}}
\put(625,68){\makebox(0,0){$32$}}
\put(625.0,812.0){\rule[-0.200pt]{0.400pt}{4.818pt}}
\put(828.0,113.0){\rule[-0.200pt]{0.400pt}{4.818pt}}
\put(828,68){\makebox(0,0){$64$}}
\put(828.0,812.0){\rule[-0.200pt]{0.400pt}{4.818pt}}
\put(1031.0,113.0){\rule[-0.200pt]{0.400pt}{4.818pt}}
\put(1031,68){\makebox(0,0){$128$}}
\put(1031.0,812.0){\rule[-0.200pt]{0.400pt}{4.818pt}}
\put(1233.0,113.0){\rule[-0.200pt]{0.400pt}{4.818pt}}
\put(1233,68){\makebox(0,0){$256$}}
\put(1233.0,812.0){\rule[-0.200pt]{0.400pt}{4.818pt}}
\put(1436.0,113.0){\rule[-0.200pt]{0.400pt}{4.818pt}}
\put(1436,68){\makebox(0,0){$512$}}
\put(1436.0,812.0){\rule[-0.200pt]{0.400pt}{4.818pt}}
\put(220.0,113.0){\rule[-0.200pt]{292.934pt}{0.400pt}}
\put(1436.0,113.0){\rule[-0.200pt]{0.400pt}{173.207pt}}
\put(220.0,832.0){\rule[-0.200pt]{292.934pt}{0.400pt}}
\put(45,472){\makebox(0,0){Time (seconds)}}
\put(828,23){\makebox(0,0){Size of side of grid}}
\put(828,877){\makebox(0,0){Iterative method timings}}
\put(220.0,113.0){\rule[-0.200pt]{0.400pt}{173.207pt}}
\put(1306,767){\makebox(0,0)[r]{'gs.table'}}
\put(1328.0,767.0){\rule[-0.200pt]{15.899pt}{0.400pt}}
\put(254,271){\usebox{\plotpoint}}
\multiput(254.00,271.58)(0.862,0.499){213}{\rule{0.789pt}{0.120pt}}
\multiput(254.00,270.17)(184.363,108.000){2}{\rule{0.394pt}{0.400pt}}
\multiput(440.00,379.58)(1.130,0.499){169}{\rule{1.002pt}{0.120pt}}
\multiput(440.00,378.17)(191.920,86.000){2}{\rule{0.501pt}{0.400pt}}
\put(1350,767){\raisebox{-.8pt}{\makebox(0,0){$\Diamond$}}}
\put(254,271){\raisebox{-.8pt}{\makebox(0,0){$\Diamond$}}}
\put(440,379){\raisebox{-.8pt}{\makebox(0,0){$\Diamond$}}}
\put(634,465){\raisebox{-.8pt}{\makebox(0,0){$\Diamond$}}}
\put(1306,722){\makebox(0,0)[r]{'mu.table'}}
\multiput(1328,722)(20.756,0.000){4}{\usebox{\plotpoint}}
\put(1394,722){\usebox{\plotpoint}}
\put(254,275){\usebox{\plotpoint}}
\multiput(254,275)(18.322,9.752){11}{\usebox{\plotpoint}}
\multiput(440,374)(18.791,8.814){10}{\usebox{\plotpoint}}
\multiput(634,465)(18.840,8.710){11}{\usebox{\plotpoint}}
\multiput(833,557)(18.927,8.517){10}{\usebox{\plotpoint}}
\multiput(1033,647)(18.656,9.096){11}{\usebox{\plotpoint}}
\put(1234,745){\usebox{\plotpoint}}
\put(1350,722){\makebox(0,0){$+$}}
\put(254,275){\makebox(0,0){$+$}}
\put(440,374){\makebox(0,0){$+$}}
\put(634,465){\makebox(0,0){$+$}}
\put(833,557){\makebox(0,0){$+$}}
\put(1033,647){\makebox(0,0){$+$}}
\put(1234,745){\makebox(0,0){$+$}}
\sbox{\plotpoint}{\rule[-0.400pt]{0.800pt}{0.800pt}}%
\put(1306,677){\makebox(0,0)[r]{'fmu.table'}}
\put(1328.0,677.0){\rule[-0.400pt]{15.899pt}{0.800pt}}
\put(254,129){\usebox{\plotpoint}}
\multiput(254.00,130.41)(3.668,0.504){45}{\rule{5.923pt}{0.121pt}}
\multiput(254.00,127.34)(173.706,26.000){2}{\rule{2.962pt}{0.800pt}}
\multiput(440.00,156.41)(1.146,0.501){163}{\rule{2.026pt}{0.121pt}}
\multiput(440.00,153.34)(189.795,85.000){2}{\rule{1.013pt}{0.800pt}}
\multiput(634.00,241.41)(1.161,0.501){165}{\rule{2.051pt}{0.121pt}}
\multiput(634.00,238.34)(194.743,86.000){2}{\rule{1.026pt}{0.800pt}}
\multiput(833.00,327.41)(1.103,0.501){175}{\rule{1.958pt}{0.121pt}}
\multiput(833.00,324.34)(195.936,91.000){2}{\rule{0.979pt}{0.800pt}}
\multiput(1033.00,418.41)(1.108,0.501){175}{\rule{1.967pt}{0.121pt}}
\multiput(1033.00,415.34)(196.917,91.000){2}{\rule{0.984pt}{0.800pt}}
\put(1350,677){\raisebox{-.8pt}{\makebox(0,0){$\Box$}}}
\put(254,129){\raisebox{-.8pt}{\makebox(0,0){$\Box$}}}
\put(440,155){\raisebox{-.8pt}{\makebox(0,0){$\Box$}}}
\put(634,240){\raisebox{-.8pt}{\makebox(0,0){$\Box$}}}
\put(833,326){\raisebox{-.8pt}{\makebox(0,0){$\Box$}}}
\put(1033,417){\raisebox{-.8pt}{\makebox(0,0){$\Box$}}}
\put(1234,508){\raisebox{-.8pt}{\makebox(0,0){$\Box$}}}
\end{picture}

    \label{fig:all-three}
  \end{center}
  \caption{\label{all three}Iterative method performances}
\end{figure}


%fmc
%plot function fit


\subsection{Picture}
\label{sec:picture}

I didn't have time to do anything fancy with graphics, an area I don't
really know much about (but wish to).  My program will generate a
simplistic VRML file, however, treating the solution field as an
\texttt{ElevationGrid}.  An example is in \texttt{dataset.data} (see
\texttt{Makefile} for options used).


\end{document}
